\chapter{World Building}

\begin{multicols}{2}

This chapter is meant only for the Mystery Weaver. It offers
guidance on how to build a game world appropriate for Mysteries
of a Broken World.

\section{Generating a World with Hexes}

All world generation stems from the hex map that the Mystery Weaver
uses to keep track of the world and the players' interaction with
it.

Each hex on the map represents an area that is 2 miles to a side.
All of the rules presented here are intended for working with hexes.
While it's possible to adapt them to work with square grids or other
systems, hexes work best.

\section{Building the Players' Freehold}

The first world-building task for any \textit{Mysteries} Mystery Weaver is
to create the freehold that the player characters call home. This
will be the players' home base, their main source of food and
information, and their cultural foundation.

All freeholds are towns with specific characteristics. They're big
enough to support a self-sufficient community, but small enough
that they're still struggling to survive. Usually this means they
have between 5,000 and 10,000 inhabitants.

Roll on the following tables to generate the freehold.

\bottomcaption{Freehold populations}
\tablefirsthead{\hline \multicolumn{1}{|c|}{\textbf{Roll}} &
											 \multicolumn{1}{c|}{\textbf{Population}} \\ \hline }
\begin{center}
{\rowcolors{3}{white}{light-gray}
\begin{xtabular}{|l|l|}
1 & 5000 \\
2 & 6000 \\
3 & 7000 \\
4 & 8000 \\
5 & 9000 \\
6 & 10000 \\
\hline
\end{xtabular}
}
\end{center}

\end{multicols}
