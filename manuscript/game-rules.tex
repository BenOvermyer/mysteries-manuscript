\chapter{Game Rules}

\begin{multicols}{2}

The objective of the game is for the players to develop interesting
stories about their characters and the world they're exploring. There
is no way to "win" the game. If a character dies, then the player can
create a new character and continue with the party's story.

While characters will gain power and possessions over time, the most
valuable progression is in their interactions with each other, with
their freehold, and with the world at large.

This chapter tells you how to play this game.

\section{The Mystery Weaver}

One player acts as the Mystery Weaver and is the referee and guide for
the game. She is responsible for building the game world, setting up
encounters, and guiding the players through the game. When a dispute
arises, the Mystery Weaver has final say.

\section{The Golden Rule}

The Golden Rule is this:

\begin{displayquote}
If the rules do not specify what happens in a given scenario,
the group chooses how to handle it.
\end{displayquote}

The Golden Rule should be invoked when the rules are unclear or simply
don't cover the scenario. Generally, choose whatever sounds the most fun.

\section{Taking Turns}

Each player takes turns describing their actions. A turn is meant to
represent a different amount of time in the game world, depending on
what the acting character is involved in. The following are the three
types of scenario that determine how long a turn is.

\textbf{Combat:} In the thick of a battle, the action is fast and furious.
Turns represent five seconds.

\textbf{Active:} When characters are in unfamiliar surroundings or
otherwise paying close attention to what's going on around them, turns
represent roughly five minutes.

\textbf{Extended:} Any scenario not covered by the first two counts as
extended time. A turn in this case is arbitrary in game time length,
and the players should agree on what it means.

\section{Turn Order}

Outside of combat, turn order is up to the players and Mystery Weaver. Just
make sure that everyone gets a turn, including the quiet players.

\section{Encounters}

As player characters move around the world, they will encounter other
characters, creatures, and monsters. These encounters rarely start out
as hostile, unless the player characters are doing something that would
put them directly at odds with the other characters.

On initially encountering someone (or something) else, the party as a
group must make a single Encounter roll to determine the other party's
disposition towards them. Roll 1d12 and consult the following table.

\bottomcaption{Encounter roll results}
\tablefirsthead{\hline \multicolumn{1}{|c|}{\textbf{Roll}} &
											 \multicolumn{1}{c|}{\textbf{Result}} \\ \hline }
\begin{center}
{\rowcolors{3}{white}{light-gray}
\begin{xtabular}{|l|l|}
1-2 & Immediate attack \\
3-5 & Hostile, but warning \\
6-10 &  Neutral \\
11-12 & Curious or friendly \\
\hline
\end{xtabular}
}
\end{center}

\section{Starting Combat}

When player characters get into a fight with someone else, we say that
"combat begins." Combat turn order takes over from normal turn order.
In combat turn order, the characters all act in order of highest Dexterity.
If two or more characters have equal Dexterity, they roll 1d6, and the
higher roller goes first. This turn order is set at the very beginning of
combat and doesn't change.

Characters controlled by the Mystery Weaver act all at once, in whatever
order the Mystery Weaver decides. They always go after all the players
in the combat turn order.

\section{Attacking}

If a character attacks another character or monster as its combat turn
action, the controlling player rolls 1d20 and compares it to the Armor
Class of the target. If the roll is higher, the attack hits and deals
damage. Otherwise, it misses, is deflected, or is dodged.

The Armor Class of an unarmored target is 9.

\section{Dealing Damage}

Weapons do 1d6 damage on a successful hit. Unarmed strikes do 1d4 damage.
This damage is subtracted from the target's current Hit Points. If those
Hit Points reach zero, then the target dies.

\section{Death and Unconsciousness}

When a character dies, his player may make a Saving Throw (if he hasn't
already used it for the day). If the Saving Throw is successful, he stays
at 1 Hit Point, but is knocked unconscious.

Monsters and other Mystery Weaver-controlled characters do not make Saving
Throws.

\section{Casting Spells}

A character spends Mana Points to cast spells. Each spell specifies how
many Mana Points it costs to cast. A spell can't be cast if the cost
is higher than the Mana Points the caster has remaining.

\section{Traps}

While exploring, characters may encounter traps. Before they run into
them, the Mystery Weaver rolls 1d20. If the number is higher than 15, then
one or more of the player characters spots the trap before it's triggered.

\textbf{Disarming traps:} A character that chooses to try and disarm a trap
rolls 1d20. If the result is lower than her Dexterity, the trap is
disarmed. Otherwise, she triggers the trap!

Traps have different effects. See the Traps appendix for more information.

\section{Saving Throws}

Saving Throws allow players to avoid dire consequences like death or
dismemberment. Once per day of in-game time, a player may make a Saving
Throw. He rolls 1d20, and if the result is 16 or higher, he succeeds.
The Mystery Weaver determines the alternative effects.

Saving Throws are most commonly used to avoid death. However, they may
also be used to avoid the worst effects of spells, traps, or other
hazards.

\section{Difficult Actions}

If a player character attempts an action that is really difficult, then
her player must roll 1d20 and try and get under the appropriate Ability
Score. The Mystery Weaver determines which Ability is the appropriate one.
If the roll is under the Ability Score, the action is successful. Otherwise,
it fails, and the Mystery Weaver determines what happens as a result.

\section{Encumbrance}

There are no mechanical limits to how much a character can carry, but the
Mystery Weaver is free to judge this for herself. Common sense applies.

\section{Ammunition}

Mundane ammunition like arrows doesn't need to be tracked. However, always
keep track of special ammunition that is hard to come by, such as enchanted
arrows. The Mystery Weaver has final say as to what counts as special ammunition.

\section{Natural Healing}

Damage is healed at a rate of 2 HP whenever the character sleeps. If he is
eating and drinking properly, he heals 4 HP instead of 2 HP. See below.

\section{Food and Water}

Player characters must eat at least twice a day, and drink water at least
once a day. Going without either food or water for more than a day takes its
toll.

If a character is eating and drinking enough, he will heal for 4 HP whenever
he sleeps instead of 2 HP.

If a character doesn't eat for at least three days, all of his ability scores
drop by 1 temporarily. Every day thereafter, they drop another 1 point. If
any of them reach 0, the character dies. A character who hasn't eaten for
longer than three days does not heal when sleeping.

If a character doesn't drink water for three days, he dies.

\section{Sleep}

If a character goes without sleep for over three days, he dies.

\section{Light}

Most dungeons and ruins are dark places with little or no light. Characters
must bring their own light with them. Many denizens of the dark fear the
light, and are particularly afraid of fire.

Lit torches illuminate a spherical area 10 feet in diameter. They last for
about an hour before burning out.

Lit lanterns illuminate the same area, but last for as long as they have oil.
A full lantern will burn for eight hours.

Campfires illuminate a spherical area 20 feet in diameter.

\section{Hex Maps and Exploration XP}

Players map out the world using paper divided into hexagonal areas. Each hex
represents an area of the world that is 2 miles to a side.

The Mystery Weaver starts the game by giving the players a hex map with their freehold
and two rings of hexes around it filled in. The rest of the world is unknown to
the player characters.

Every time the party maps a new hex, each member of the party gains 50 XP.

\end{multicols}
