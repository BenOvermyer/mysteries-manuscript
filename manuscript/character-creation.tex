\chapter{Character Creation}

\begin{multicols}{2}

This chapter details the rules for creating the character you will play.
If your Mystery Weaver is providing characters for you, you can skip this
chapter.

\section{Summary of Character Creation}

\begin{enumerate}
	\item Roll ability scores
	\item Choose a bloodline
	\item Choose an archetype
	\item Choose a background
	\item Choose an alignment
	\item Roll for afflictions
	\item Choose spells, if applicable
	\item Roll for mana points, if applicable
	\item Roll for hit points
	\item Roll for starting money
	\item Fill out details, as desired
\end{enumerate}

\section{Ability Scores}

For each of the six scores, roll 3d6.

You may swap two scores once after you've finished rolling, OR you can reroll
all six scores once.

\textbf{Strength:} A measure of your physical might. Governs how much damage you
do in melee combat.

\textbf{Dexterity:} Your agility and deftness. Affects your ranged combat damage
and when you act in combat.

\textbf{Constitution:} A reflection of your physical and mental endurance. This
affects your total Hit Points as well as your resistances.

\textbf{Wisdom:} This is a measure of your common sense and understanding of the
ways of the world. It is vital for adherents of the divine.

\textbf{Intelligence:} This is how clever and learned you are. It affects arcane
spellcasting and how many languages you know.

\textbf{Charisma:} This is your social ability, force of personality, and general
attractiveness. It affects how many NPC retainers you can have, as well as
leadership and persuasion.

\section{Bloodline}

Choose one of the following as your bloodline. The bloodlines found in any given Freehold
vary greatly.

\subsection{Dwarf}

\textbf{Adulthood:} 50 years

\textbf{Maximum Age:} 300 years

Dwarves are short and stocky. Both men and women can grow prodigious beards.
They have a natural affinity for stone and can navigate underground as if they
could see for miles.

\subsection{Elf}

\textbf{Adulthood:} 50 years

\textbf{Maximum Age:} Unknown

Elves are thin, tall, and elegant. Their ears are pointed. They live eternally
unless slain. Elves are agile and quick and have an affinity for magic. No living
elf is old enough to remember the Breaking.

\subsection{Halfling}

\textbf{Adulthood:} 20 years

\textbf{Maximum Age:} 200 years

Halflings are short, but normally proportioned for their size. They have
large, hairy feet and usually go barefoot. Many seem to have a knack for
moving quietly.

\subsection{Human}

\textbf{Adulthood:} 18 years

\textbf{Maximum Age:} 100 years

Humans vary greatly in appearance and demeanor. Their skin ranges in color
from deepest black to palest white, with no one hue being more common than
the others.

\subsection{Minotaur}

\textbf{Adulthood:} 20 years

\textbf{Maximum Age:} 150 years

Minotaurs are large, bulky humanoids with the head and horns of a bull. They
tower over most other bloodlines.

\subsection{Sandman}

\textbf{Adulthood:} 20 years

\textbf{Maximum Age:} 100 years

Sandmen are humanoids that appear to be made of sand. Their eyes are glassy and
their skin is yellow or brown. They feel no heat or cold.

\section{Archetype}

Choose one of the following. This is your character's primary profession and
the source of your main abilities. It also gives you your Hit Die.

\subsection{Alchemist}

\textbf{Hit Die:} d4

\textbf{Armor Allowed:} Leather, Padded

\textbf{Weapons Allowed:} Any one-handed

Alchemists are brewers of potions and transmuters of materials. They have a
greater understanding of the laws of the natural world than most others. An
alchemist may craft potions.

\subsection{Berserker}

\textbf{Hit Die:} d10

\textbf{Armor Allowed:} None

\textbf{Weapons Allowed:} Any

Berserkers are warriors who have made a pact with the spirit of Fury. During
battle they may give themselves over to Fury, gaining great strength and speed.
However, they have no control over their actions until Fury releases them.

\subsection{Bard}

\textbf{Hit Die:} d4

\textbf{Mana Die:} d6

\textbf{Armor Allowed:} Leather, Padded

\textbf{Weapons Allowed:} Any one-handed

Bards are musicians who weave magic into their songs. They might use musical
instruments or rely only on their voice. Their magic is subtle and moving.
Bards may learn spells from the Enchantment list.

\subsection{Demon-Caller}

\textbf{Hit Die:} d4

\textbf{Mana Die:} d8

\textbf{Armor Allowed:} None

\textbf{Weapons Allowed:} One-handed melee

Demon-callers are aggressive magic-users who derive their power from dark entities.
They have made pacts to gain power, and must serve the ends of their patron.
A demon-caller may learn spells from the Demonic list.

\subsection{Man-At-Arms}

\textbf{Hit Die:} d6

\textbf{Armor Allowed:} Any

\textbf{Weapons Allowed:} Any

Men-at-arms are professional soldiers. Their craft is war. They are proficient
in the use of all weapons and armor. A man-at-arms may choose one weapon to
specialize in, and from that point forward, gain a +1 to hit with that weapon.

\subsection{Monk}

\textbf{Hit Die:} d8

\textbf{Armor Allowed:} Padded

\textbf{Weapons Allowed:} None

Monks are warriors of a particular church. They eschew the use of weapons and
armor in favor of perfecting unarmed combat. Each church has a different style
of fighting, usually themed after the tenets of their patron god.

\subsection{Paladin}

\textbf{Hit Die:} d8

\textbf{Armor Allowed:} Any

\textbf{Weapons Allowed:} Any

Paladins are zealous soldiers of a particular church. They adhere strictly to
a code of conduct, even when that adherence puts them or others at risk. A
paladin may heal another fully by touching them and invoking their deity once per day.

\subsection{Priest}

\textbf{Hit Die:} d6

\textbf{Mana Die:} d6

\textbf{Armor Allowed:} Leather, Padded, Chain

\textbf{Weapons Allowed:} Any blunt

Priests are the voices of the gods. They adhere rigidly to the rituals and
practices of their church, and in exchange, are given the ability to work
miracles. Priests may use spells from the Divine list. They may also Turn
Undead.

\subsection{Ranger}

\textbf{Hit Die:} d6

\textbf{Armor Allowed:} Leather, Padded

\textbf{Weapons Allowed:} Any ranged, one-handed melee

Rangers are woodland archers and hunters. Their proficiency in stealth is
unmatched. They know the woods more than any other mortal. A ranger gains +1 to
hit with bows.

\subsection{Channeler}

\textbf{Hit Die:} d4

\textbf{Mana Die:} d12

\textbf{Armor Allowed:} None

\textbf{Weapons Allowed:} One-handed melee

Channelers are arcane spellcasters that do not rely on book learning for their
magic. Instead, they may innately cast spells. Channelers only get spells from
the Elemental list. They may not learn spells from any external source. Instead,
every time a channeler gains a level, he chooses one new spell from the Elemental
list.

\subsection{Thief}

\textbf{Hit Die:} d6

\textbf{Armor Allowed:} Leather, Padded

\textbf{Weapons Allowed:} Any one-handed

Thieves are rogues and highwaymen. They are skilled at sneaking into places and
stealing whatever valuables may lie therein. A thief excels at picking locks,
disarming or setting traps, and is a master of all forms of stealth.

\subsection{Tree-Speaker}

\textbf{Hit Die:} d4

\textbf{Mana Die:} d8

\textbf{Armor Allowed:} Leather, Padded

\textbf{Weapons Allowed:} Any one-handed

Tree-speakers are spellcasters in tune with the natural world. They revere nature
and work to protect it and maintain its balance. Tree-speakers may use spells from
the Nature list.

\subsection{Wizard}

\textbf{Hit Die:} d4

\textbf{Mana Die:} d8

\textbf{Armor Allowed:} None

\textbf{Weapons Allowed:} One-handed melee

Wizards are arcane scholars and practitioners of High Magic. They study the inner
workings of magic and seek to unlock its deepest secrets. Wizards may learn
spells from any list other than Divine or Demonic.

\section{Background}

The following reflects a profession or calling that your character had prior
to picking up the mantle of adventurer, or that their parents wanted them to
pick up, or something along those lines. It represents your general area of
knowledge, as well as some specific skills you might apply when not off
adventuring.

You may choose one, or roll 1d100 on the table.

\bottomcaption{List of backgrounds}
\tablefirsthead{\hline \multicolumn{1}{|c|}{\textbf{Roll}} &
											 \multicolumn{1}{c|}{\textbf{Background}} \\ \hline }
\begin{center}
{\rowcolors{3}{white}{light-gray}
\begin{xtabular}{|l|l|}
01-02 & Apothecary \\
03-04 & Armorer \\
05-06 & Astronomer \\
07-08 & Baker \\
09-10 & Barber \\
11-12 & Barrister \\
13-14 & Blacksmith \\
15-16 & Bookbinder \\
17-18 & Bowyer \\
19-20 & Brewer \\
21-22 & Bricklayer \\
23-24 & Butler \\
25-26 & Candlemaker \\
27-28 & Carpenter \\
29-30 & Cartographer \\
31-32 & Chaplain \\
33-34 & Cook \\
35-36 & Courtesan \\
37-38 & Dyer \\
39-40 & Engraver \\
41-42 & Falconer \\
43-44 & Farmer \\
45-46 & Fisherman \\
47-48 & Fortune Teller \\
49-50 & Furrier \\
51-52 & Gardener \\
53-54 & Glassblower \\
55-56 & Gravedigger \\
57-58 & Herald \\
59-60 & Horse Trainer \\
61-62 & Hunter \\
63-64 & Innkeeper \\
65-66 & Jester \\
67-68 & Leatherworker \\
69-70 & Merchant \\
71-72 & Moneylender \\
73-74 & Musician \\
75-76 & Painter \\
77-78 & Poet \\
79-80 & Potter \\
81-82 & Rat Catcher \\
83-84 & Sailor \\
85-86 & Scout \\
87-88 & Scribe \\
89-90 & Sculptor \\
91-92 & Shipwright \\
93-94 & Shoemaker \\
95-96 & Squire \\
97-98 & Town Guard \\
99-00 & Trapper \\
\hline
\end{xtabular}
}
\end{center}

\section{Alignment}

Your character's alignment reflects what primal force influences their life.
It may manifest subtly, such as in influencing your character's decisions.
It may also manifest dramatically, such as a monster with the same alignment
appearing in the area. Such major manifestations are rare, however.

Characters can only use magical items that share their own alignment, or have
no alignment.

\textbf{Equilibrium:} The primal force of Equilibrium seeks to limit all other
forces. Those who align with Equilibrium are just, merciful, or callous.

\textbf{Chaos:} The primal force of Chaos seeks continual renewal and change. Those
who align with Chaos are impulsive, brash, or mercurial.

\textbf{Destiny:} The primal force of Destiny seeks continuity and predictability.
Those who align with Destiny are imperious, fatalistic, or resilient.

\textbf{Void:} The primal force of Void seeks emptiness, clarity, and purity. Those
who align with Void are thoughtful, stern, or aloof.

\section{Afflictions}

Afflictions are random mutations caused either by mutant parentage or by
encountering a place of wild magic left over from the Breaking. Regardless
of whether the Affliction has a positive effect or not, all those who have
an Affliction are shunned by normal society. They are regarded as the
Afflicted.

Your character has a chance to have an Affliction at character creation. Roll
1d100. If the result is 3 or less, roll 1d100 on the following table.

\bottomcaption{List of afflictions}
\tablefirsthead{\hline \multicolumn{1}{|c|}{\textbf{Roll}} &
											 \multicolumn{1}{c|}{\textbf{Affliction}} \\ \hline }
\begin{center}
{\rowcolors{3}{white}{light-gray}
\begin{xtabular}{|l|l|}
01-03 & Albinism \\
04-08 & Allergies \\
09-10 & Black and White Vision \\
11-13 & Claws \\
14-19 & Color Blindness \\
20-21 & Dwarfism \\
22-25 & Enhanced Sense of Smell \\
26-28 & Extra Fingers \\
29-33 & Fangs \\
34-35 & Forked Tongue \\
36-39 & Functional Gills \\
40-41 & Fur \\
42-46 & Gigantism \\
47-50 & Hairless \\
51-53 & Horns \\
54-55 & Occasional Seizures \\
56-60 & Odd Hair Color \\
61-67 & Oddly Colored Skin \\
68-69 & Permanent Boils \\
70-73 & Scaly Skin \\
74-75 & Strong Body Odor \\
76-78 & Tail \\
79-81 & Third Eye \\
82-84 & Thorny Skin \\
85-96 & Unnatural Eyes \\
97-98 & Webbed Feet \\
99-99 & Weird Voice \\
00-00 & Roll Again Twice \\
\hline
\end{xtabular}
}
\end{center}

\section{Spells}

If your character has a spellcasting archetype, then choose one spell from the
appropriate spell lists. This is the spell that your character knows at
the start of their journey.

\section{Mana Points}

If your character has a spellcasting archetype, then roll the Mana Die of your
archetype. If the result is a 1, you may reroll once. If your Intelligence
is higher than 12, add 1 to the result. This is your maximum Mana Points.

\section{Hit Points}

Look up the Hit Die of your archetype. Roll one of those dice. If the result is a
1 or a 2, you may reroll once. If your Constitution is higher than 12, add 1
to the result. This is your maximum Hit Points.

\section{Starting Money}

Your character begins the game with 3d6 x 10 silver coins. You can use this
money to buy starting equipment. See the Equipment chapter for a list of
things that you can buy. Adventurers typically are given this money by the
freehold they belong to, but the money may come from other sources.

\section{Character Details}

You may wish to write down a few extra details about your character, though
this is not required. Some things you might want to think about are:

\begin{itemize}
	\item Age
	\item Gender
	\item Weight
	\item Height
	\item Hair Color and Style
	\item Eye Color
	\item Skin Color
	\item Body Shape
	\item Family
	\item Hobbies
	\item Motivations
	\item Core Beliefs
\end{itemize}

\section{Advancement}

Your character will gain Experience Points for bringing treasure and knowledge
back from the wilds. Experience Points (or XP for short) are only gained once
your character returns what they've found to the rulership of their Freehold.

Once you have earned enough XP, you will gain a level. Once you do so, you roll
your archetype Hit Die and add the result to your maximum Hit Points. If your
Constitution score is higher than a 12, you add 1 to this total each time you
level up.

If you have a spellcasting archetype, the maximum level of spell that you can cast
increases by 1 each time you level up. So, if you are level 3, you can cast up
to (and including) level 3 spells. Also, roll your archetype Mana Die and add the
result to your maximum Mana Points. If your Intelligence score is higher than a
12, you add 1 to this total each time you level up.

Levels are gained every 1,000 XP.

\section{Learning Spells}

New spells are only gained by learning them from artifacts found in the ruins
of the old world. Most often, these will be spell scrolls or books. Another
spellcaster who knows the spell you seek to learn may be able to teach it to
you, but you must meet the level and spell list restrictions in order to
learn it.

Learning a new spell takes a number of weeks equal to the spell's level. The
caster must spend that time studying and practicing.

\end{multicols}
