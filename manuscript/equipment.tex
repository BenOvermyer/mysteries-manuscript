\chapter{Equipment}

\begin{multicols}{2}

The following is a list of all the equipment that
characters might buy or otherwise acquire through
their adventures.

Note: because freeholds are isolated and have little
in the way of resources, it's unlikely that any given
freehold will have everything in these lists.

In particular, things that require a lot of artisanal
skill to create—like full plate—are going to be
extremely rare.

\section{Costs and Currency}

After the breaking of the world, there was no central
authority to control the value of coinage. As such, no
one currency is considered to be the standard.

Much coinage still exists in the world, but it's judged
by its weight and not its appearance.

For simplicity's sake, all costs in this chapter are
given in gold coins (gc), silver coins (sc), copper
coins (cc), or pieces of eight (pe).

A gold coin is worth ten silver coins.

A silver coin is worth ten copper coins.

So, a gold coin is worth a hundred copper coins.

Also, copper coins are often cut into pieces for smaller
transactions. A single coin will be cut into eight pieces,
and so "pieces of eight" are used for things that cost
less than a copper coin.

Most people do not have a steady income. However, the
average village craftsman will likely earn roughly
8 gc over the course of a year in a freehold.

\end{multicols}

\section{Armor}

\bottomcaption{Types of armor}
\tablefirsthead{\hline \multicolumn{1}{|c|}{\textbf{Type}} &
											 \multicolumn{1}{c|}{\textbf{Armor Class}} &
											 \multicolumn{1}{c|}{\textbf{Cost}} \\ \hline }
\begin{center}
{\rowcolors{3}{white}{light-gray}
\begin{xtabular}{|l|l|l|}
Padded & 11 & 2 gc \\
Leather & 12 & 5 gc \\
Chain & 13 & 12 gc \\
Splint & 14 & 17 gc \\
Scale & 15 & 24 gc \\
Breastplate & 16 & 50 gc \\
Full Plate & 17 & 150 gc \\
\hline
\end{xtabular}
}
\end{center}

\section{Shields}

Shields increase the user's Armor Class by 1 if worn.
Note: tower shields are not meant to be worn, but rather
to be used as mobile walls.

\bottomcaption{Types of shields}
\tablefirsthead{\hline \multicolumn{1}{|c|}{\textbf{Type}} &
											 \multicolumn{1}{c|}{\textbf{Cost}} \\ \hline }
\begin{center}
{\rowcolors{3}{white}{light-gray}
\begin{xtabular}{|l|l|}
Buckler & 2 gc \\
Heater & 4 gc \\
Tower & 20 gc \\
\hline
\end{xtabular}
}
\end{center}

\section{Weapons}

\bottomcaption{List of afflictions}
\tablefirsthead{\hline \multicolumn{1}{|c|}{\textbf{Weapon}} &
											 \multicolumn{1}{c|}{\textbf{Hands}} &
											 \multicolumn{1}{c|}{\textbf{Type}} &
											 \multicolumn{1}{c|}{\textbf{Melee/Ranged}} &
											 \multicolumn{1}{c|}{\textbf{Cost}} \\ \hline }
\begin{center}
{\rowcolors{3}{white}{light-gray}
\begin{xtabular}{|l|l|l|l|l|}
Two-handed Axe & 2 H & Slashing Melee & 7 gc \\
Hand axe & 1 H & Slashing Melee & 4 gc \\
Short Sword & 1 H & Slashing Melee & 7 gc \\
Longsword & 1 H & Slashing Melee & 10 gc \\
Two-handed Sword 2 H & Slashing Melee & 15 gc \\
Staff & 2 H & Blunt & Melee & 5 sc \\
Dagger & 1 H & Piercing Melee & 1 sc \\
Pick & 1 H & Piercing Melee & 1 gc \\
Morningstar & 1 H & Blunt & Melee & 5 gc \\
Mace & 1 H & Blunt & Melee & 3 gc \\
Maul & 2 H & Blunt & Melee & 6 gc \\
Warhammer & 1 H & Blunt & Melee & 5 gc \\
Trident & 2 H & Piercing Melee & 5 gc \\
Spear & 2 H & Piercing Melee & 2 gc \\
Polearm & 1 H & Piercing Melee & 7 gc \\
Flail & 1 H & Blunt & Melee & 4 gc \\
Whip & 1 H & Blunt & Melee & 6 sc \\
Sling & 1 H & Blunt & Ranged & 5 cc \\
Shortbow & 2 H & N/A & Ranged & 3 gc \\
Longbow & 2 H & N/A & Ranged & 8 gc \\
Crossbow & 2 H & N/A & Ranged & 10 gc \\
Arrows x10 & N/A & Piercing N/A & 5 sc \\
Crossbow Bolts x5 N/A & Piercing N/A & 1 gc \\
\hline
\end{xtabular}
}
\end{center}

\textbf{Note:} You can use bows as a blunt weapon, but they will break
the first time you do this. After that, they are useless until
repaired.

\section{Tools}

These tools are readily available in most freeholds.

\bottomcaption{Types of tools}
\tablefirsthead{\hline \multicolumn{1}{|c|}{\textbf{Type}} &
											 \multicolumn{1}{c|}{\textbf{Cost}} \\ \hline }
\begin{center}
{\rowcolors{3}{white}{light-gray}
\begin{xtabular}{|l|l|}
Anvil & 1 gc \\
Armorer's tools & 13 gc \\
Augur & 2 pe \\
Backpack 5 sc \\
Bellows & 1 gc \\
Hammer & 1 cc \\
Hand mirror & 1 gc \\
Iron spike & 1 pe \\
Rope (50ft) & 1 cc \\
Sack, large & 5 cc \\
Sack, small & 1 cc \\
Shovel & 2 pe \\
Spade & 1 pe \\
Torch & 1 cc \\
Vise & 6 sc \\
\hline
\end{xtabular}
}
\end{center}

\section{Food and Drink}

Most common items found in inns and country houses are
listed here. More exotic foods and drinks are covered
in the world-building chapter.

\bottomcaption{Types of food and drink}
\tablefirsthead{\hline \multicolumn{1}{|c|}{\textbf{Type}} &
											 \multicolumn{1}{c|}{\textbf{Cost}} \\ \hline }
\begin{center}
{\rowcolors{3}{white}{light-gray}
\begin{xtabular}{|l|l|}
Ale (good), 1 barrel & 4 sc \\
Ale (good), 1 mug & 2 pe \\
Ale (poor), 1 barrel & 2 sc \\
Ale (poor), 1 mug & 1 pe \\
Egg & 1 pe \\
Fish (fried) & 2 pe \\
Fish (salted) & 4 pe \\
Handful of sugar & 1 sc \\
Haunch of meat & 1 cc \\
Loaf of bread & 1 pe \\
Meat stew & 5 pe \\
Side of Bacon & 6 cc \\
Wedge of cheese & 1 pe \\
Wine (good), 1 bottle & 9 pe \\
Wine (poor), 1 bottle & 3 pe \\
\hline
\end{xtabular}
}
\end{center}

\section{Livestock}

While adventuring or traveling for long periods of time,
it may be helpful to have a ready meat source available.

\bottomcaption{Types of livestock}
\tablefirsthead{\hline \multicolumn{1}{|c|}{\textbf{Type}} &
											 \multicolumn{1}{c|}{\textbf{Cost}} \\ \hline }
\begin{center}
{\rowcolors{3}{white}{light-gray}
\begin{xtabular}{|l|l|}
Chicken & 4 pe \\
Cow & 4 sc \\
Goat & 1 sc \\
Goose & 6 pe \\
Sheep & 2 sc \\
\hline
\end{xtabular}
}
\end{center}

\section{Horses}

Horses are valuable. A single quarterhorse is likely the most
expensive thing a typical farmer owns. In the world after the
Breaking, horses that are formally trained for battle are
almost unheard of.

\bottomcaption{Types of horses}
\tablefirsthead{\hline \multicolumn{1}{|c|}{\textbf{Type}} &
											 \multicolumn{1}{c|}{\textbf{Cost}} \\ \hline }
\begin{center}
{\rowcolors{3}{white}{light-gray}
\begin{xtabular}{|l|l|}
Quarterhorse & 1 gc \\
Riding horse & 10 gc \\
Warhorse & 1,000 gc \\
\hline
\end{xtabular}
}
\end{center}

\section{Housing}

Building a house requires land ownership or rent, if you're
building it near a freehold. If you're building in the wilderness,
good luck to you.

Renting a room for a night at an inn varies, but is commonly
less than a copper coin.

\bottomcaption{Types of housing}
\tablefirsthead{\hline \multicolumn{1}{|c|}{\textbf{Type}} &
											 \multicolumn{1}{c|}{\textbf{Cost}} \\ \hline }
\begin{center}
{\rowcolors{3}{white}{light-gray}
\begin{xtabular}{|l|l|}
Cottage & 10 gc \\
Craftman's house & 40 gc \\
Merchant's house & 120 gc \\
Noble's house & 500 gc \\
\hline
\end{xtabular}
}
\end{center}
